This essay has attempted to briefly discuss what CA may mean for the future of humanity. It has firstly discussed the five potential benefits and five risks of CA. It than moved on to apply three normative ethical theories to analyse the morality of CA; the analysis found that utilitarianism can be used to justify CA if implemented correctly, duty ethics may face some difficulties with CA due to its rigid nature and finally that virtue ethics can be guiding principle for a positive implementation and application of CA in society. However, the ethical discussion revealed some pitfalls associated with each ethical theory is present when implemented for CA, thus a pragmatic application of all three theories need to be made to obtain a just and fair ethical code for CA.

Finally some recommendations were made about the humanity's approach towards CA. It is the belief of the authors that CA is something that we will need to deal with; conservatism will not only hinder progress, it will handicap our ability to control the development of CA and implement policy and framework to regulate to lead to the greatest human good. Emphasis was put into both a gradual and thoughtful development process as well as a slow exposure of this technology to the public. These would be pivotal in the way CA pans out in the future because having a slow and careful development process in parallel with a slow step wise public exposure of the technology will allow the developers to better understand the consequence of the technology and leave time for ancillary institutional and policy frameworks to develop. Better understanding of the technological consequences as well as an accommodating framework will guide this technology so that it can lead humanity to a utopia and prevent it from causing a dystopian implosion.
