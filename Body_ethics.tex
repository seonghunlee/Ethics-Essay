\subsection{Normative ethical perspective on CA}

Ethics is defined as "the systematic reflection of morality"\cite[ch. 3.2]{Ethics_textbook}; morality is the notion of right and wrong as expressed by the thoughts, actions and decisions of an individual or a collective. The introduction of CA technology in society leads to a vast array of ethical issues. In this essay the perspective of normative ethics will be applied to look at how ethical theories may consider CA to be moral or immoral; the three major ethical theories applied will be Utilitarianism, Duty ethics and Virtue ethics.

\subsubsection{Utilitarianism}

"In utilitarianism, actions are judged by the amount of pleasure and pain they bring about. The action that brings the greatest happiness for the greatest number should be chosen." \cite[ch. 3.7]{Ethics_textbook} Further addition that need to be made to refine this theory is the Freedom principle, which states that people are free to pursue their own pleasures as long as it does not hinder the freedom of others. 

For CA to be acceptable according to utilitarianism the net benefits need to outweigh the net risks. This is something that will be hard, if not impossible, to establish with scientific studies. The effect of the Collingridge dilemma \cite{collingridge1980social} is extensively pronounced in the case of CA research and implementation due to both the potential unknown capability and the extensively strong effect it may have on humanity's development. Notions of Constructive Technology Assessment (CTA) \cite{schot1992constructive,schot1997past} can be possibilities to guide its progress. There are also issues with the freedom principle and the notion of distributive justice that maybe problematic with the applications of CA. 

However, in the end it must be realized that greater utility is what has driven technological progress; CA will lead to a net increase in human potential. If too many social costs are not incurred and an equitable distribution of CA can be achieved while maintaining the rights and liberties of the individuals, it can easily be imagined how CA would lead to a better life for humanity's majority.

%maybe make this part shorter

\subsubsection{Duty ethics}

Duty ethics judges actions to be right or wrong based on how much they agree to a moral norm. These norms are described by duty ethics as "Categorical imperative"; there are two primary categorical imperatives in duty ethics; namely the "Universality principle" and the "Reciprocity principle". The universality principle states "Act only on that maxim which you can at the same time will that it should become a universal law"\cite[ch. 3.8]{Ethics_textbook}. The reciprocity principle states "Act as to treat humanity, whether in your own person or in that of any other, in every case as an end, never as means only" \cite[ch. 3.8]{Ethics_textbook}.

A big issue with the application of duty ethics, more specifically the universality principle, will be the "slippery slope argument" \cite{van1991slippery,resnik1994debunking}; this argument states that if permitting a good action X creates precedent for permitting a bad action Y, X should not be permitted. Due to the nature of CA it is unlikely that a rigid framework of universal laws will be applicable. More problems can be seen if we apply the second categorical imperative to the analysis; it can be applied to CA in two opposite ways. Firstly, it can be argued that CA causes human beings to lose humanity (i.e. become more machine like) and thus may encourage the tendency of treating them like means to achieve certain ends. On the other hand, if one adopts the view that CA will be a means that humanity will use to aid or improve itself with negligible risk to itself, than this imperative is no longer violated.

It can be understood from the above discussion that the application of duty ethics to CA may lead to more problems than solutions.

\subsubsection{Virtue ethics}

Virtue ethics focuses on the characteristics of the people involved. It is different from the other ethical theories in the sense that it ignores the moral quality of actions and argues that qualities of the actor will lead to moral actions. This theory encourages the nurturing and developing of virtues that lead to the realization of an "individual's unique human potential"; this leads to what is called "a good life" or "eudaimonia" \cite[ch. 3.9]{Ethics_textbook}. The good life is not necessarily a state that brings the greatest pleasure to an individual neither is it one that can be achieved by adhering to a strict code of norms.

In light of this theory, CA seems to find a reasonable standing; however, it must be noted that according to virtue ethics, only the kind of CA that can allow a person to flourish and improve upon her virtues will be accepted and encouraged. Any other kind of CA, for example one which makes people more violent, less able to make decisions, or makes an athlete less agile, will all be discouraged. Virtue ethics espouses an idea of "practical wisdom" for the individual which allows her to make the right moral decision when faced with ambiguity. The application of such practical wisdom will be required by humanity as a whole to guide it through the many pitfalls of CA.

Although useful for the rightful and beneficial application of CA in humanity, it has a big problem. It makes regulation and legislation difficult for it is hard to judge an action based on the vice or virtues of the people involved.