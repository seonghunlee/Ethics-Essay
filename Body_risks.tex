\subsection{Risks of cybernetic augmentation}

%will have to add citations for many things

Cybernetic augmentation may also have negative consequences. As mentioned in the previous section, the examples of technologies discussed can be imagined to be a double edged sword; it may help humanity but it may as easily harm it. These possible of risks of cybernetic enhancement are: detrimental restructuring of human life, loss of humanity, hazards to health, decreased safety and security and finally unhappiness.

(1) Detrimental restructuring of human life \\

Cybernetic augmentation will most likely make many of the current social institutions and regulations meaningless; they would require them to be redefined and rethought accounting for the effect of new augmentations. For example, new laws would need to be defined for physical cybernetic augmentations that may allow human beings to run at really high speeds; academic performance may become meaningless if cognitive implants capable to storing or recalling information like a computer become available {\bf MAYBE REFER TO BOSTROM AND ROACHE}. Cybernetic enhancements may drastically increase the life expectancy of people which may require us to change the way approach reproduction. A great deal of jobs will be lost because fewer people would be required to perform jobs; this may lead to a global economic crisis. It will most likely intensify the already problematic situation of social stratification in the world; if augmentations are expensive the rich will get an even greater edge over the poor. Different types of enhanced humans may create their own closed communities and due to their enhancements maybe assigned different (possibly greater) rights over non-enhanced humans.

%Necessity to change many regulations and laws, as well as to create ones
%Need for rethinking of most institutions; e.g. having memory chips would nullify most academic tests we have now.
%economic consequence
%-Increased life expectancy has side effects: over population, increased social-welfare cost, youth unemployment, etc.
%-Employment issue (i.e. few people replacing many people's job)
%social stratification
%-Different classes of humanity with different rights (rethinking of basic human rights)
%-Affordability leading to human segregation and social startification (violation of utilitarian ethics)
%-Different classes of humanity with different rights (rethinking of basic human rights)
%-Communicatability between two types of sensory enhanced human
%-Inequality issue - The rich is most likely to get an earlier (or even monopolistic) access to such technology and gain an unfair advantage of cybernetic augmentation.

(2) Loss of humanity \\

Critics of cybernetic augmentation (and human enhancement in general) often present what is referred to as the argument of naturalness {\bf CITE TO PEOPLE ANDY MIAH CITED IN PG190(24)} against augmentation; there is an intuitive belief that changing human biology would be detrimental to what it is to be human {\bf CITE TO PEOPLE 25QA}. Supplanting these arguments are, among others, the fear of 'playing god' and the notion that a human being is forged through hardships she overcomes; if augmentation make those hardships easy to obtain, they may lose their meaning.

%risk to humanity and way of life
%Human dignity - loss of "humanness"
%Manipulation of memory which maybe argued as fundamental to the fabric of human character formation (violation of virtue ethics; barrier to eudamonia)

(3) Hazards to health \\

There are many potential health hazards to cybernetic augmentation. Due to its invasive nature, there would be great amount to risk to human life both in the research and development phase of such technology, because it will need to be tested on human beings. Assuming the technology is developed and implemented, there would be the perpetual risks of having a malfunctioning cybernetic device inside the human body; besides the possibility of malfunction there would be risks of hazardous interaction with the environment.

%Short-term risks that the lack of knowledge and maturity of the technology may harm the bodies and decrease the convenience, as opposed to the original intentions.
%Unknown psychological side-effect
%Risk to humanbody

(4) Decreased safety and security \\

Safety may also decrease since augmented criminals will be better criminals. The controlling of cybernetic augmentation may lead to the formation of black markets which will contribute further to crime. Wars may become more dangerous since both there will be deadlier soldiers and more capable generals leading them. With cognitive augmentation there is the risk of loss of autonomy and privacy; there maybe the risk of thought monitoring by criminals or governments. In an extreme case, ones mind maybe hacked or their personality significantly altered without their consent.

%more capable criminals; fierce wars. more ways to exploit
%Increased sense of insecurity (Analogous to gun control issue)
%Military application can be problematic in view of ethics. Can we justify the use of cybernetic augmentation to produce a stronger army? What is the soldiers' point of view, and what is the significance for the humanity?
%Loss of autonomy and privacy
%-Mind reading
%-Potential hacking of other human beings
%-Misuse by the government; e.g. with interrogation, monitoring of citizens thoughts, or altering their personality

(5) Unhappiness \\

Besides the above mentioned risks and their consequential unhappiness, cybernetic augmentation may cause unhappiness intrinsically. If augmentations are impossible or riskier to remove than implement, than there would be the chance that one would limit the possibilities of her life by choosing a specific type of augmentation. A highly prolonged life may be a source for unhappiness; one may find herself to be bored of life and find nothing meaningful left to pursue. If ignorance truly is bliss, cognitive enhancement that would lead to greater amounts of intellect may end up leading to unhappiness.




%{\color{red} This bullet-list is made by Seong - These are the risks of physical CA}
%
%
%\begin{itemize}
%	\item Short-term risks that the lack of knowledge and maturity of the technology may harm the bodies and decrease the convenience, as opposed to the original intentions.
%	\item Unknown psychological side-effect
%	\item Increased life expectancy has side effects: over population, increased social-welfare cost, youth unemployment, etc.
%	\item Increased sense of insecurity (Analogous to gun control issue)
%	\item Employment issue (i.e. few people replacing many people's job)
%	\item Military application can be problematic in view of ethics. Can we justify the use of cybernetic augmentation to produce a stronger army? What is the soldiers' point of view, and what is the significance for the humanity?
%	\item Necessity to change many regulations and laws, as well as to create ones
%	\item Inequality issue - The rich is most likely to get an earlier (or even monopolistic) access to such technology and gain an unfair advantage of cybernetic augmentation.
%	\item Human dignity - loss of "humanness"
%	
%\end{itemize}
%
%{\color{red} The following bullet-list is made by Manan - These are the risks of cognitive CA that you wrote in the first draft}
%
%\begin{itemize}
%	\item Risk to humanbody, humanity and way of life
%	\item Affordability leading to human segregation and social startification (violation of utilitarian ethics)
%	\item Different classes of humanity with different rights (rethinking of basic human rights)
%	\item Communicatability between two types of sensory enhanced human
%	\item Need for rethinking of most institutions; e.g. having memory chips would nullify most academic tests we have now.
%	\item Manipulation of memory which maybe argued as fundamental to the fabric of human character formation (violation of virtue ethics; barrier to eudamonia)
%	\item Misuse of this technology; potential harm that humanity can unleash on itself with these enhancements. This may express itself as the following (violation of duty ethics):
%	\begin{itemize}
%		\item Mind reading
%		\item Potential hacking of other human beings
%		\item Misuse by the government; e.g. with interrogation, monitoring of citizens thoughts, or altering their personality
%	\end{itemize}
%\end{itemize}