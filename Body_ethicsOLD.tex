\subsection{Application of ethical theories towards cybernetic augmentation}
%The main ethical issue that the introduction of cybernetic augmentation technology imposes on the society can be viewed in two different (but closely related) approaches: (1) an approach to deal with the ethical aspects of the technical risks by analyzing whether or not such risks are acceptable, and (2) an approach of duty ethics or virtue ethics which judges the act of cybernetic augmentation itself by directly relating to the moral values, such as human dignity. In the following sections, the two approaches are elaborated so as to discuss the ethical implications of cybernetic augmentation.

%the way i see it approach 1 is mills utilitarianism with more details. because the most relevant one seems a risk-benefit analysis.

The introduction of cybernetic augmentation technology in society leads to a vast array of ethical issues, from which a select few will be discussed in this essay. First cybernetic augmentation will be analysed with respect to the three major ethical theories; namely Utilitarianism, Duty ethics and Virtue ethics. Following this three other ethical issues that will arise due to cybernetic augmentation will be discussed. These issues are the Collinridge dilemma, the ethical principles of engineers on cybernetic augmentation development and the effect on responsibility and blameworthiness of individuals due to augmentations.

\subsubsection{Normative ethical perspective on cybernetic augmentation}

\paragraph{Utilitarianism}

According to \cite{Ethics_textbook}, a risk can be considered morally acceptable only after considering the following aspects:
\begin{enumerate}
	\item The degree of informed consent with the risk
	\item The degree to which the benefits weigh up against the risks
	\item The availability of alternatives with a lower risk, and
	\item The degree to which risks an advantages are justly distributed (among the people)
\end{enumerate}


First of all, the principle of informed consent states that the potential risks and benefits must be fully informed to the people who might be influenced by the activity, and if they all agree to take the risks, then the risk can be considered morally acceptable. This is based on the freedom principle which respects the moral autonomy of the individuals and suggests that everyone is free to strive for his/her own pleasure, as long as there is no harm to others \cite{Ethics_textbook}. This alone however cannot become the sole criterion for judging the acceptability of risks because it is practically impossible to get consents from every individual who is influenced by the technology. As shown in the previous section, the potential risks of cybernetic augmentation exist not only towards the direct users, but also on the general society in the long term. It therefore becomes impractical to ask for consents from everyone. Moreover, considering the fact that the cybernetic augmentation technology is a newly emerging field, it is very difficult to fully comprehend the potential risks beforehand. 

The second consideration is the use of risk-cost-benefit analysis. This is based on the idea of utilitarianism (i.e. a type of consequentialism based on the utility principle which strives for the greatest happiness for the greatest number). Such approach can be useful as it can give us an intuitive idea on the overall impacts of the activity. It is for this reason that the potential benefits and risks are discussed in the previous section. However, one should be careful to avoid the fallacy of pricing (i.e. expressing every value in monetary terms) and realize that the implication of the cybernetic augmentation technology is multidimensional.

The third criterion for the acceptability of the risk are rather subtle in this case, as this paper focuses on the evaluation of the technology itself, rather than an engineering problem which strives to solve an existing problem. 

The last consideration to be taken into account is whether the risk and benefits are evenly distributed among the people. As discussed in the previous section, it is predictable that the rich will enjoy the benefits of cybernetic augmentation more, thereby causing an inequality issue. Nevertheless, it is also assumable that the users of cybernetic augmentation in the earlier phase of the development would take more risks than the users  in the later phase of the development where the application of technology has improved with more maturity.

All in all, it can be reasoned that the acceptability of the risk of cybernetic augmentation should not be judged based only on each criterion that is mentioned, but also through the continuous reflection and multidimensional considerations during the process of development and deployment of the technology.

\paragraph{Duty ethics}

Whereas the analysis of the moral acceptability of the risks appears to be rather existentialistic, the approach from a duty ethics or virtue ethics allows us to consider a different aspect of the issue - Is cybernetic augmentation virtuous? Does it deteriorate moral values, such as human dignity? Is it the right thing to do? 

\paragraph{Virtue ethics}



\subsubsection{Other ethical issues regarding cybernetic augmentations}

\paragraph{Collinridge dilemma} \\

\paragraph{Ethical principles of engineers} \\

\paragraph{Changes to notions of responsibility} \\
%\subsubsection{Analysis of the acceptability of the risks}


%\subsubsection{Duty ethics or Virtue ethics approach}


If 

%Instead, one should be aware of the fact that such ethical issues are often very abstract and subjective, as it directly deals with the moral values of the people which is reflected on the zeitgeist of the society. 

%One may dispute against this perspective by using an analogy to other currently available technology (e.g. "The cars enabled us to move faster, and the computers allowed for better management of information. No one says cars and computers damage our human dignity. The same should go for cybernetic augmentation." However, such argument is again subject to disputes as the cybernetic augmentation takes an intrusive form and becomes "as if it was a natural part of the human body". Therefore, the analogy between the cybernetic augmentation and currently existing technologies is not perfectly justifiable.

% % % NOTES

%# technology assessment, collinridge dilemma,CTA; pg 32 ch1
% maybe some kind of talk about responsibility ch1 & 8
%# Ethhical principles for engineers in a global environment (Luegenbiehl,2010); pg 55 ch2
% quote no harm principle and freedom principle when used subsubsection of acceptability of risk; pg74-5 ch3 ethics book
% use concepts of distributive justice and marginal utility; pg 76 ch3, ethics book
% use the concept of eudamonia from aristotles vitue ethics; pg 84 ch3, ethics book
%# issue of the slipperly slope argument as in some paper i read; issue of similarity to the following quote if we choose for CA 'Vaughan’s analysis of the Challenger disaster illustrates a more general point: decisions – also incremental and implicit ones – tend to commit us to certain courses of actions and frame subsequent decisions (Darley, 1996).' pg 145 ethics book
% ethics during approach to design i.e. responsibility of researchers and designers; problem of 'radical design concept in CA, this maybe part of risks aswell. ch 6;
% technological mediation, moralizing of technology ch 7. moralizing of tech mentioned in benefits part
% strong presence of both uncertainity and ignorance in CA, the ignorance can be more conseuquential. this point maybe more appropriate in risks part ch 8




