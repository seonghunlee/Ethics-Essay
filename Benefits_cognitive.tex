There are three primary types of cognitive enhancements that can be attained through cybernetics. 1. improvement of thinking capacity; 2. improvement of sensory capacities; 3. accomodation for a mind machine interface. Current innovations have only began to scratch the surface of devices that may lead to the second or third type of enhancements. Most of the devices being researched are not intended for enhancement rather aiding people with existing mental or physical disablities. 

As an example of a device that effects sensory capacity one may consider the cochlear implants \bf{NEEDS A CITATION} which directly interface with the nerves connecting the ear and the brain to give severly deaf people a sensation of sound. Devices such as the Argus II \bf{NEEDS A CITATION} and Alpha IMS \bf{NEEDS A CITATION} are implants placed directly within the retina of a person suffering from vision loss due to inability of the retial cells to detect light. These implants are stimulated either by processing the image externally in the case of Argus II or by the directly by the light entering the eye, which are than converted to electronic impulses by the implants and sent to the brain. Researchers at MIT are working on a device \bf{NEEDS A CITATION}, which is at a very early stage of development, that plans to fully bypass the eye and send impulses directly to the brain to give its user some kind of vision.

Examples of existing devices for interfacing between the brain and an external object range from simple, wearable and off the shelf EEG devices such as Emotiv \bf{NEEDS A CITATION} and iFocusBand \bf{NEEDS A CITATION} to expermental devices that physically interfaces with the brain using electrodes such as BrainGate \bf{NEEDS A CITATION}. While the former is mostly used for recreational purposes, the latter is being researched and developed to give people physically disabled some means of communicating with the world.


\bf{copied from wiki} 

Ethical considerations

Important ethical, legal and societal issues related to brain-computer interfacing are:[97][98][99][100]

    conceptual issues (researchers disagree over what is and what is not a brain-computer interface),[100]
    obtaining informed consent from people who have difficulty communicating,
    risk/benefit analysis,
    shared responsibility of BCI teams (e.g. how to ensure that responsible group decisions can be made),
    the consequences of BCI technology for the quality of life of patients and their families,
    side-effects (e.g. neurofeedback of sensorimotor rhythm training is reported to affect sleep quality),
    personal responsibility and its possible constraints (e.g. who is responsible for erroneous actions with a neuroprosthesis),
    issues concerning personality and personhood and its possible alteration,
    therapeutic applications and their possible exceedance,
    questions of research ethics that arise when progressing from animal experimentation to application in human subjects,
    mind-reading and privacy,
    mind-control,
    use of the technology in advanced interrogation techniques by governmental authorities,
    selective enhancement and social stratification, and
    communication to the media.

Clausen stated in 2009 that “BCIs pose ethical challenges, but these are conceptually similar to those that bioethicists have addressed for other realms of therapy”.[97] Moreover, he suggests that bioethics is well-prepared to deal with the issues that arise with BCI technologies. Haselager and colleagues[98] pointed out that expectations of BCI efficacy and value play a great role in ethical analysis and the way BCI scientists should approach media. Furthermore, standard protocols can be implemented to ensure ethically sound informed-consent procedures with locked-in patients.

Researchers are well aware that sound ethical guidelines, appropriately moderated enthusiasm in media coverage and education about BCI systems will be of utmost importance for the societal acceptance of this technology. Thus, recently more effort is made inside the BCI community to create consensus on ethical guidelines for BCI research, development and dissemination.[100]


\bf{end copied from wiki} 
