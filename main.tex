\input{preamble}
\pretitle{\begin{center} \fontsize{20}{20} \usefont{OT1}{phv}{b}{n} \selectfont}
	\title{Cybernetic Augmentation - a Key to Utopia or Dystopia?}			  
	% Title of your article goes here
	
	\posttitle{\vspace{7pt} \par {\Large WM0324LR Ethics Essay 2014-2015} \end{center} \vspace{0.1cm}}



\preauthor{\begin{center} \lineskip 0.1cm }
	\author{Seong Hun Lee 4145623, Manan Siddiquee 4185692}										% Author names go here
	\postauthor{\end{center}} % Change your group name!

 \date{}																	% Do not show date in \maketitle command


%%% Begin document
\begin{document}
	\twocolumn[\begin{@twocolumnfalse}
		\maketitle
		
		\begin{abstract}
		\noindent																% Remove default indentation
		\textbf{ \\ Cybernetic augmentation (in short, CA) technology is a newly emerging field of human engineering. In this essay, the authors analyse and disuss the potential ethical implications of the technology by (1) identifying the plausible benefits and risks from the social, economical, and technological point of view, and (2) applying the normative ethical theories. Major potential benefits of CA are identified as: ease of hardships, improvement of health and survivavility, creation of new opportunities, enhanced sense of safety and welfare, increased efficiency and productivity, and greater happiness. On the other hand, major potential risks could be: detrimental restructuring of human life, loss of humanity, hazard to health, decreased safety and security, and greater unhappiness. Next to the risk-benefit analysis, the ethical aspect of the intoduction of the technology is contemplated in terms of three normative ethical theories - utilitarianism, duty ethics, and virtue ethics. Discussions showed the complexity of the issue, and emphasized the neccessity of a pragmatic application of all three theories in order to obtain a solid ethical code for CA. Finally, the authors provide recommendations which suggest the importnace of a gradual and careful development process in prallel with a continuous public discussion for the better comprehension and appreciation of the technology.\\}
		\end{abstract} 
		\vspace{0.5cm}
		
	\end{@twocolumnfalse}]

	%% The sections
	\section{Introduction} 
	%Our biggest defining characteristic as human beings is our intellect. Needless to say we are not the strongest, the fastest or the most resilient to change that nature has to offer. Still we populate most of the biomes of this world and consciously define our lives and our habitats like no other animal on Earth. All of this in spite of our frailty compared to what nature has to offer, and all of this due to the only advantage that nature gave us; our ability to think better than any other animal that we know of.

%The source of the prosperity that humans have achieved in the last few millennium is the ability to make tools to complement ourselves; we have made tools to adapt the environment more habitable, ensured our food supply, increased our strength and dexterity, widened our knowledge and improved our ability to carry out mental activities. However, after the  millennium of progress we are approaching a new dimension of technology; a dimension which some believe will be our salvation while others would easily relate to the legends of Icarus, Pandora’s box and Prometheus’ fire. The dimension referred to is the prospect of actively and intrusively improving the human mind and body. Developments in fields such as electronics, nanotechnology, robotics, cybernetics, information technology, neurotechnology, genetic engineering and pharmacology, among others, are enabling a new field of technology labeled by some as ‘Human Engineering’ to emerge.

Humanity's ability to make tools to complement itself can arguably be the most important contributor to its success over the last few millennia; we have made tools to adapt the environment to our needs, ensure our food supply, increase our strength and dexterity, widen our knowledge and improve our ability to carry out mental activities. However, after millennia of progress we are approaching a new dimension of technology; a dimension which some believe will be our salvation while others think it will be our undoing. The dimension referred to is the prospect of actively and intrusively improving the human mind and body. Developments in fields such as electronics, nanotechnology, robotics, cybernetics, information technology, neurotechnology, genetic engineering and pharmacology, among others, are enabling a new field of technology labelled by some as "Human Engineering" to emerge.

The authors of this paper realize that human engineering is a very broad and trans-disciplinary field, and it would take volumes to analyse its ethical implications; thus this paper attempts to narrow the field by only looking at the ethical implications of one form of ‘Human Engineering’; namely cybernetic augmentation of human beings (henceforth CA). For the purpose of this paper CA is defined as the following: any (1) electro-mechanical addition to the human body that (2) becomes a natural part of the human body and that (3) improves the performance of an individual beyond normal human capacities. Needless to say genetic or chemical forms of human augmentation are external to the purview of CA. Furthermore, the scope of CA as defined for this paper deals with the enhancement of human capabilities "beyond the statistically normal standard". Therefore it does not consider the therapeutic use of the technology (e.g. prosthetic limbs for the amputees, artificial eyes/ears for the blind/deaf, memory chip for the patients suffering from Alzheimer's disease, etc.). Although there are complications for making this divide between therapy and enhancement, they will also be considered external to the scope of this paper.

The paper is structured as follows; the first part of the essay looks at the effects of CA on society; following this an ethical theories are used to analyse the morality of augmentation. After looking at the effects and their ethical implications, recommendations are made to support the implementation of CA on society. Finally, the conclusions are drawn regarding the viewpoints and approaches towards CA, its ethics and its effects on the future of humanity.
	\section{Analysis}
    To explore the potential ethical implications of cybernetic augmentation in a more structured manner, a distinction is made between the two types of cybernetic augmentations: 
\begin{enumerate}
	\item Augmentations to enhance the user's physical ability
	\item Augmentations to enhance the cognitive/sensory ability.
\end{enumerate}

In this essay, physical cybernetic augmentation is defined as any kind of cybernetic augmentation that enhances the physical (i.e. of the body) qualities of a human being. Examples of physical improvement include among others a reconstruction/replacement of certain body parts with enhancement of physical abilities, or modification of the body to adapt to diverse environments (e.g. giving human beings gills and webbed feet to adapt for life underwater or improving the lungs and skin to tolerate different types of atmospheres).

As for the cognitive cybernetic augmentation, there are mainly three primary types: 1. improvement of thinking capacity; 2. improvement of sensory capacities; 3. accommodation for a interfacing with the external world. Current innovations have only began to scratch the surface of devices that may lead to the second or third type of enhancements. Most of the devices being researched are not intended for enhancement rather aiding people with existing mental or physical disabilities; however, one need only apply a little imagination to see how the application of these technologies to people without illness or disabilities may lead to enhancements of their functionalities.

%For example, if one's arm is replaced by a robotic arm which gives the user a superior arm with better strength and agility, then it is seen as an augmentation which enhances the physical ability. On the other hand, if a computer chip is implemented in one's brain to increase the user's memory capacity, or an electronic lens is augmented to enable the user to see infrared rays, then it is considered as an augmentation which enhances the cognitive/sensory ability of the user. 

With such a distinction in mind, let us analyse and discuss the ethical implications of cybernetic augmentation. Before we start, it is important to ask ourselves "What is meant by an ethical implication?". According to the Longman Dictionary of Contemporary English, an implication means "a possible future effect or result of an action, event, decision, etc" \cite{Longman_dic}. Therefore, an ethical implication would mean the possible future effect or result with regard to associated moral values and principles of morality. For analysis, this section is divided into three main parts: 

\begin{enumerate}
	\item Potential benefits of cybernetic augmentation
	\item Potential risks of cybernetic augmentation
	\item Application of ethical theories 
\end{enumerate}

The first two parts provide an insight into the possible social, economical, and technological influence of the development and application of cybernetic augmentation technology, as well as its potential direct impact on users' physical and psychological state. Finally, the ethical implications of the technology are analysed and discussed by relating to the existing ethical theories and models.


%\begin{enumerate}
%	\item Potential benefits of cybernetic augmentation
%	\item Potential risks of cybernetic augmentation
%	\item Ethical implications of cybernetic augmentation
%\end{enumerate}
%The first two parts provide an insight into the possible social, economical, and technological influence of the development and application of cybernetic augmentation technology, as well as its potential direct impact on users' physical and psychological state. Subsequently, the ethical implications of the technology are analyzed and discussed by relating to the existing ethical theories and models.

%% Manan(14-12-2014 1636): The division of ethical implications to potential benefits and risks is flawed i think. ethics has to do with whats right and wrong, not what is beneficial or harmful per se. a beneficial thing can be wrong based on values and vice versa.

    \subsection{Benefits of cybernetic augmentation}

There can be numerous potential benefits when the cybernetic augmentation is implemented to enhance the user's physical and/or cognitive ability. In this essay, some of the main benefits of cybernetic augmentation are considered and discussed. These are: better well-being, higher chances of survival, increased efficiency and productivity, self-defense, and further development of technology. Of course, there will be many other unmentioned potential benefits (and risks), but it should be noted that our intention is to give an overview and "food for thought" that is useful and adequate to comprehend the general implications, rather than to give you an exhaustive list of every possibility in the future. 

{\bf (1) Better well-being of the user} \\ 
One way that the cybernetic augmentation can be used to achieve the sense of better well-being is by enabling the users to carry out their daily actions in a more convenient way. For example, with augmented arms or legs, one may never have to struggle when lifting or carrying things. Also, the level of pain might be controlled in those augmented organs - for example, when you touch something very cold or hot beyond a certain threshold level, then it may regulate the synaptic information to your nerve system that you do not feel the pain you would have felt if you were not augmented. Such technology of physical cybernetic augmentation can give the user more control over their bodies, and thus reducing the level of inconvenience and stress. Furthermore, one can also consider the use of sensory/cognitive augmentation, such as an augmented eye which enables the user to see clearly during night with little light; this may be done by modifying or improving visual prosthesis devices \bf{CITE TO VISUAL PROSTHESES FOR BLIND}. Another kind of cognitive augmentation can be a computer chip that can be implemented in the brain to enhance the memory capability \bf{CITE TO NEURAL PROSTHESIS BERGER ET AL}. Such technology of cognitive cybernetic augmentation can lead to the better perception and management of the daily-life information.

%Another view with respect to the well-being is that cybernetic augmentation can positively affect the users not only physically, but also psychologically, as the extra abilities can plausibly help them gain higher self-esteem and confidence. In particular, for those people with the psychological complex about certain parts of their body, or abilities which they feel inferior themselves, the cybernetic augmentation could be used to overcome such complexes and give them more sense of happiness and better well-being.

{\bf (2) Higher chances of survival and adaptation to the new environment} \\
Since cybernetic enhancements will enhance the mind and the body of the person in question, it can be argued that their chances of survival will increase. A stronger, smarter person will be able to handle himself better in case of dangers such as natural disasters or accidents. In addition, the cybernetic augmentation can help the users maintain or improve their health by incorporating advanced medical technology. For example, the research is ongoing on the development of the nanorobots which are designed to navigate through our bodies' blood vessels, detect the cancerous cell, and kill it \cite{nanorobot}. Through the technology of cybernetic augmentation, such nanorobots can monitor our body more comprehensively, and perform medical tasks more quickly and efficiently at an early stage. Imagining further into the future, cybernetic augmentations may give us the ability to inhabit currently uninhabitable environments; for example, underwater or other planets with hazardous environments.

%The enhanced strengths and intelligence gained by the cybernetic augmentation are very likely to increase the chance of the user's survival in the wild nature amongst the beasts of prey or other hazardous creatures as the technology can possibly allow him/her to evolve into a stronger and more protected species. For the similar reason, the chances of survival will become generally higher for the cybernetically enhanced person when encountered disasters or catastrophic accidents. More futuristically, the cybernetic augmentation can exploited to allow the users to live a sustainable life in those places that are currently considered uninhabitable - for example, underwater, or other planets with hazardous environments. Such extension of the habitable territory for the humans can be viewed as beneficial in many ways. One obvious benefit would be the alleviation of the problems caused by the overpopulation.

{\bf (3) Increased efficiency and productivity}\\
Enhanced physical capability of the workers by means of cybernetic augmentation is most likely to increase the efficiency and productivity of humanity. Humanity will be able to better optimize and specialize its manpower, meaning as a sum effect we will get more done. Augmentations can be doctored to suit the professions of individuals; for example, construction workers can get physical augmentation that make them stronger, miners may consider improving their lungs to filter out harmful pollutants, businessmen can consider neural implants that will keep his mind connected to important information networks.

%This would be a good news to the employers, since they can either reduce the labour cost by employing less number of people to do the same job, or increase the profits with the same number of employees because the they would have enhanced efficiency and productivity, thanks to the cybernetic augmentation. Another economical benefit would be the new creation of jobs; the tasks which are considered as currently impossible or very difficult may become practicable when the workforce with enhanced ability are engaged. Despite such economic benefits, one should however still keep in mind the side-effect that the currently existing jobs may require less people once the qualities of workforces are drastically enhanced through cybernetic augmentation, and therefore can lead to the short-term rise of the unemployment rate.

{\bf (4) Self-defense and military application}\\
One may claim that cybernetic augmentation which enhances the physical ability would lead to more security as it can be used as means of self-defense. However, this is indeed subjected to the dispute that it can be used also as a weapon to attack others, which is analogous to the current debate regarding the gun control issue in U.S. Nevertheless, it is hard to deny that the cybernetic augmentation can be used to enhance the level of self-defense of the user, compared to the non-users of cybernetic augmentation.

Such advantage of cybernetic augmentation can be most extensively exploited in the field of military application; the soldiers with high-level cybernetic augmentation can gain enhanced physical combat ability.

{\bf (5) Further development of technology}\\
Once the physical cybernetic augmentation becomes an active trend of the society, there will be more initiatives for the further research, development and application of the technology in the fields of not only in cybernetic augmentation, but also in other fields of technology and industries. For instance, the design of the interfaces of many electronic gadgets or machines can change into a more efficient form (e.g. semi-automated guns that can be attached to the part of an augmented arm, or infrared monitors/screens when the visible frequency of our eyes can be controlled using an advanced electronic lens, etc)



    \subsection{Risks of cybernetic augmentation}
{\color{red} This bullet-list is made by Seong - These are the risks of physical CA}


\begin{itemize}
	\item Short-term risks that the lack of knowledge and maturity of the technology may harm the bodies and decrease the convenience, as opposed to the original intentions.
	\item Unknown psychological side-effect
	\item Increased life expectancy has side effects: over population, increased social-welfare cost, youth unemployment, etc.
	\item Increased sense of insecurity (Analogous to gun control issue)
	\item Employment issue (i.e. few people replacing many people's job)
	\item Military application can be problematic in view of ethics. Can we justify the use of cybernetic augmentation to produce a stronger army? What is the soldiers' point of view, and what is the significance for the humanity?
	\item Necessity to change many regulations and laws, as well as to create ones
	\item Inequality issue - The rich is most likely to get an earlier (or even monopolistic) access to such technology and gain an unfair advantage of cybernetic augmentation.
	\item Human dignity - loss of "humanness"
	
\end{itemize}

{\color{red} The following bullet-list is made by Manan - These are the risks of cognitive CA that you wrote in the first draft}

\begin{itemize}
	\item Risk to humanbody, humanity and way of life
	\item Affordability leading to human segregation and social startification (violation of utilitarian ethics)
	%\item Different classes of humanity with different rights (rethinking of basic human rights)
	\item Communicatability between two types of sensory enhanced human
	\item Need for rethinking of most institutions; e.g. having memory chips would nullify most academic tests we have now.
	\item Manipulation of memory which maybe argued as fundamental to the fabric of human character formation (violation of virtue ethics; barrier to eudamonia)
	\item Misuse of this technology; potential harm that humanity can unleash on itself with these enhancements. This may express itself as the following (violation of duty ethics):
	\begin{itemize}
		\item Mind reading
		\item Potential hacking of other human beings
		\item Misuse by the government; e.g. with interrogation, monitoring of citizens thoughts, or altering their personality
	\end{itemize}
\end{itemize}
    \subsection{Ethical implications of CA}
This is the Ethical implications of CA
 %    \input{physicalCA}
 %    \input{cognitiveCA}
    \section{Authors' Opinions and Recommendations}
    Despite a multitude of identified technical risks and ethical concerns, the authors of this paper claim that the emergence and development of the CA should not be simply avoided, but should rather be encouraged in such a manner that the moral autonomy of those who are willing to utilize the technology is respected while the possible risks are striven to be minimized. Moreover, we argue that an excessively conservative view which tries to indiscreetly restrict the entire human augmentation technology itself will not only hinder the gateway to the great benefits of the technology, but also limit the possibility to develop a more encompassing social, economical, technological, and ethical framework which would help us better comprehend the implications of the related technologies and allow us to deal with the risks more effectively.
%\vspace{-2.5cm}

As a way of utilizing the technology while minimizing the technical risks and ethical concerns, the authors propose the following recommendations:

%\vspace{-2.5cm}
\begin{itemize}
	\item Active encouragement of the application of CA technology for therapeutic purpose (e.g. prosthetic limbs for the amputees illustrated in Figure \ref{prosthetic_arm}, and visual prostheses for the blind illustrated in Figure \ref{prosthetic_eye}). Since such kind of use of the technology is much less controversial, it is recommendable that the development of the augmentation technology be focused on these medical fields for the time being. Once the technology gains more maturity and people become more familiar (and knowledgeable) with it, we can then have more meaningful discussions concerning the acceptability of the CA technology that is intended to enhance the human ability "beyond the normal level".
	\item Active encouragement of the development of non-permanent/wearable cybernetic devices, starting from the smallest scale. If the intrusive nature of the CA is removed, there will be much less technical risks and ethical concerns. Therefore, it can be a great stepping-stone for the introduction of the CA technology for similar reasons as in the previous item. 
	\item A discussion as to whether the CA is a key to Utopia or Dystopia can be meaningful as long as the discussion actively continues throughout the development and deployment of the technology, whilst refraining from making hasty conclusions (i.e. slippery slope fallacy). Note that CA is a very broad technology. Therefore, instead of judging the final possible outcome of the entire concept of the technology, one should rather try to find specific fields and aspects in which the technology can be applied with the highest acceptability. 
\end{itemize}

\begin{figure}[]
	\centering
		\includegraphics[width = 8cm]{prothesis_arm}
	\caption{Prosthetic arm developed by DARPA. An artificial arm is connected to the user's existing muscles and nerves\cite{prosthetic_arm}}
	\label{prosthetic_arm}
\end{figure}

\begin{figure}[]
	\centering
	\includegraphics[width = 6cm]{prothesis_eye}
	\caption{Prosthetic retinal implant concept illustration \cite{retinal_implant}}
	\label{prosthetic_eye}
\end{figure}
    \section{Conclusions}
    This essay has attempted to briefly discuss what CA may mean for the future of humanity. It has firstly discussed the five potential benefits and five risks of CA. It than moved on to apply three normative ethical theories to analyse the morality of CA; the analysis found that utilitarianism can be used to justify CA if implemented correctly, duty ethics may face some difficulties with CA due to its rigid nature and finally that virtue ethics can be guiding principle for a positive implementation and application of CA in society. However, the ethical discussion revealed some pitfalls associated with each ethical theory is present when implemented for CA, thus a pragmatic application of all three theories need to be made to obtain a just and fair ethical code for CA.

Finally some recommendations were made about the humanity's approach towards CA. It is the belief of the authors that CA is something that we will need to deal with; conservatism will not only hinder progress, it will handicap our ability to control the development of CA and implement policy and framework to regulate to lead to the greatest human good. Emphasis was put into both a gradual and thoughtful development process as well as a slow exposure of this technology to the public. These would be pivotal in the way CA pans out in the future because having a slow and careful development process in parallel with a slow step wise public exposure of the technology will allow the developers to better understand the consequence of the technology and leave time for ancillary institutional and policy frameworks to develop. Better understanding of the technological consequences as well as an accommodating framework will guide this technology so 


\bibliographystyle{plain}
\bibliography{mybib}
	
\end{document}

