\subsection{Benefits of cybernetic augmentation}

There can be numerous potential benefits when the cybernetic augmentation is implemented to enhance the user's physical and/or cognitive ability. In this essay, some of the main benefits of cybernetic augmentation are considered and discussed. These are: better well-being, higher chances of survival, increased efficiency and productivity, self-defense, and further development of technology. Of course, there will be many other unmentioned potential benefits (and risks), but it should be noted that our intention is to give an overview and "food for thought" that is useful and adequate to comprehend the general implications, rather than to give you an exhaustive list of every possibility in the future. 

{\bf (1) Better well-being of the user} \\
One way that the cybernetic augmentation can be used to achieve the sense of better well-being is by enabling the users to carry out their daily actions in a more convenient way. For example, with augmented arms or legs, one may never have to struggle when lifting or carrying things. Also, the level of pain might be controlled in those augmented organs - for example, when you touch something very cold or hot beyond a certain threshold level, then it may regulate the synaptic information to your nerve system that you do not feel the pain you would have felt if you were not augmented. Such technology of physical cybernetic augmentation can give the user more control over their bodies, and thus reducing the level of inconvenience and stress. Furthermore, one can also consider the use of sensory/cognitive augmentation, such as an augmented eye which enables the user to see clearly during night with little light, or a computer chip that can be implemented in the brain to enhance the memory capability. Such technology of cognitive cybernetic augmentation can lead to the better perception and management of the daily-life information.

In addition, the cybernetic augmentation can help the users maintain or improve their health by incorporating advanced medical technology. For example, the research is ongoing on the development of the nanorobots which are designed to navigate through our bodies' blood vessels, detect the cancerous cell, and kill it \cite{nanorobot}. Through the technology of cybernetic augmentation, such nanorobots can monitor our body more comprehensively, and perform medical tasks more quickly and efficiently at an early stage. 

Another view with respect to the well-being is that cybernetic augmentation can positively affect the users not only physically, but also psychologically, as the extra abilities can plausibly help them gain higher self-esteem and confidence. In particular, for those people with the psychological complex about certain parts of their body, or abilities which they feel inferior themselves, the cybernetic augmentation could be used to overcome such complexes and give them more sense of happiness and better well-being.

{\bf (2) Higher chances of survival and adaption to the new environment} \\
The enhanced strengths and intelligence gained by the cybernetic augmentation are very likely to increase the chance of the user's survival in the wild nature amongst the beasts of prey or other hazardous creatures as the technology can possibly allow him/her to evolve into a stronger and more protected species. For the similar reason, the chances of survival will become generally higher for the cybernetically enhanced person when encountered disasters or catastrophic accidents. More futuristically, the cybernetic augmentation can exploited to allow the users to live a sustainable life in those places that are currently considered uninhabitable - for example, underwater, or other planets with hazardous environments. Such extension of the habitable territory for the humans can be viewed as beneficial in many ways. One obvious benefit would be the alleviation of the problems caused by the overpopulation.

{\bf (3) Increased efficiency and productivity}\\
An enhanced physical capability of the workers by means of cybernetic augmentation is most likely to increase the efficiency and productivity of the work and industry. This would be a good news to the employers, since they can either reduce the labor cost by employing less number of people to do the same job, or increase the profits with the same number of employees because the they would have enhanced efficiency and productivity, thanks to the cybernetic augmentation. Another economical benefit would be the new creation of jobs; the tasks which are considered as currently impossible or very difficult may become practicable when the workforce with enhanced ability are engaged. Despite such economic benefits, one should however still keep in mind the side-effect that the currently existing jobs may require less people once the qualities of workforces are drastically enhanced through cybernetic augmentation, and therefore can lead to the short-term rise of the unemployment rate.

{\bf (4) Self-defense and military application}\\
One may claim that cybernetic augmentation which enhances the physical ability would lead to more security as it can be used as means of self-defense. However, this is indeed subjected to the dispute that it can be used also as a weapon to attack others, which is analogous to the current debate regarding the gun control issue in U.S. Nevertheless, it is hard to deny that the cybernetic augmentation can be used to enhance the level of self-defense of the user, compared to the non-users of cybernetic augmentation.

Such advantage of cybernetic augmentation can be most extensively exploited in the field of military application; the soldiers with high-level cybernetic augmentation can gain enhanced physical combat ability.

{\bf (5) Further development of technology}\\
Once the physical cybernetic augmentation becomes an active trend of the society, there will be more initiatives for the further research, development and application of the technology in the fields of not only in cybernetic augmentation, but also in other fields of technology and industries. For instance, the design of the interfaces of many electronic gadgets or machines can change into a more efficient form (e.g. semi-automated guns that can be attached to the part of an augmented arm, or infrared monitors/screens when the visible frequency of our eyes can be controlled using an advanced electronic lens, etc)


