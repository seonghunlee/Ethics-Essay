To explore the potential ethical implications of cybernetic augmentation in a more structured manner, a distinction is made between the two types of cybernetic augmentations: 
\begin{enumerate}
	\item Augmentations to enhance the user's physical ability
	\item Augmentations to enhance the cognitive ability.
\end{enumerate}
For example, if one's arm is replaced by a robotic arm which gives the user a superior arm with better strength and agility, then it is seen as an augmentation which enhances the physical ability. On the other hand, if a computer chip is implemented in one's brain to increase the user's memory capacity, then it is considered as an augmentation which enhances the cognitive ability of the user. 

With such a distinction in mind, let us analyze and discuss the ethical implications of cybernetic augmentation. Before we start, it is important to ask ourselves "What is meant by an ethical implication?". According to the Longman Dictionary of Contemporary English, an implication means "a possible future effect or result of an action, event, decision, etc" \cite{Longman_dic}. Therefore, an ethical implication would mean the possible future effect or result with regard to associated moral values and principles of morality. For analysis, this can be viewed as divided into two main aspects: namely, the potential benefits and risks. Therefore, in this section, the potential benefits and risks of the cybernetic augmentation are contemplated and discussed.

% Manan(14-12-2014 1636): The division of ethical implications to potential benefits and risks is flawed i think. ethics has to do with whats right and wrong, not what is beneficial or harmful per se. a beneficial thing can be wrong based on values and vice versa.
