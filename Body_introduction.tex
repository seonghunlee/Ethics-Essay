To explore the potential ethical implications of cybernetic augmentation in a more structured manner, a distinction is made between the two types of cybernetic augmentations: 
\begin{enumerate}
	\item Augmentations to enhance the user's physical ability
	\item Augmentations to enhance the cognitive/sensory ability.
\end{enumerate}

In this essay, physical cybernetic augmentation is defined as any kind of cybernetic augmentation that enhances the physical (i.e. of the body) qualities of a human being. Examples of physical improvement include among others a reconstruction/replacement of certain body parts with enhancement of physical abilities, or modification of the body to adapt to diverse environments (e.g. giving human beings gills and webbed feet to adapt for life underwater or improving the lungs and skin to tolerate different types of atmospheres).

As for the cognitive cybernetic augmentation, there are mainly three primary types: 1. improvement of thinking capacity; 2. improvement of sensory capacities; 3. accommodation for a interfacing with the external world. Current innovations have only began to scratch the surface of devices that may lead to the second or third type of enhancements. Most of the devices being researched are not intended for enhancement rather aiding people with existing mental or physical disabilities; however, one need only apply a little imagination to see how the application of these technologies to people without illness or disabilities may lead to enhancements of their functionalities.

%For example, if one's arm is replaced by a robotic arm which gives the user a superior arm with better strength and agility, then it is seen as an augmentation which enhances the physical ability. On the other hand, if a computer chip is implemented in one's brain to increase the user's memory capacity, or an electronic lens is augmented to enable the user to see infrared rays, then it is considered as an augmentation which enhances the cognitive/sensory ability of the user. 

With such a distinction in mind, let us analyse and discuss the ethical implications of cybernetic augmentation. Before we start, it is important to ask ourselves "What is meant by an ethical implication?". According to the Longman Dictionary of Contemporary English, an implication means "a possible future effect or result of an action, event, decision, etc" \cite{Longman_dic}. Therefore, an ethical implication would mean the possible future effect or result with regard to associated moral values and principles of morality. For analysis, this section is divided into three main parts: 

\begin{enumerate}
	\item Potential benefits of cybernetic augmentation
	\item Potential risks of cybernetic augmentation
	\item Application of ethical theories 
\end{enumerate}

The first two parts provide an insight into the possible social, economical, and technological influence of the development and application of cybernetic augmentation technology, as well as its potential direct impact on users' physical and psychological state. Finally, the ethical implications of the technology are analysed and discussed by relating to the existing ethical theories and models.


%\begin{enumerate}
%	\item Potential benefits of cybernetic augmentation
%	\item Potential risks of cybernetic augmentation
%	\item Ethical implications of cybernetic augmentation
%\end{enumerate}
%The first two parts provide an insight into the possible social, economical, and technological influence of the development and application of cybernetic augmentation technology, as well as its potential direct impact on users' physical and psychological state. Subsequently, the ethical implications of the technology are analyzed and discussed by relating to the existing ethical theories and models.

%% Manan(14-12-2014 1636): The division of ethical implications to potential benefits and risks is flawed i think. ethics has to do with whats right and wrong, not what is beneficial or harmful per se. a beneficial thing can be wrong based on values and vice versa.
