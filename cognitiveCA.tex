\subsection{Cognitive Cybernetic Augmentation}

There are three primary types of cognitive enhancements that can be attained through cybernetics. 1. improvement of thinking capacity; 2. improvement of sensory capacities; 3. accommodation for a interfacing with the external world. Current innovations have only began to scratch the surface of devices that may lead to the second or third type of enhancements. Most of the devices being researched are not intended for enhancement rather aiding people with existing mental or physical disabilities; however, one need only apply a little imagination to see how the application of these technologies to people without illness or disabilities may lead to enhancements of their functionalities.

{\bf Available technology}

Currently no form of cybernetic technology exists that can supplement the thinking capacity of the bearer. However some experiments on animals have had promising results. The Prosthetic Neuronal Memory Silicon Chips designed by Theodore Berger {\bf C} have been demonstrated to be able to restore memories in rats and monkeys. Although he has not been able to form long term memories with his chip, he believes that his research will eventually lead to devices capalble of doing so. He hopes these these devices will be able to help people with Alzheimer's disease, stroke or brain injury that suffer from memory loss.

Cochlear implants {\bf C} are devices which directly interface with the nerves connecting the ear and the brain to give severly deaf people a sensation of sound; these devices are examples of cybernetic devices augmenting sound sensations. Devices such as the Argus II {\bf C} and Alpha IMS {\bf C} are implants placed directly within the retina of a person suffering from vision loss due to inability of the retial cells to detect light. These implants are stimulated either by processing the image externally in the case of Argus II or by the directly by the light entering the eye, which are than converted to electronic impulses by the implants and sent to the brain. 

Examples of existing devices for interfacing between the brain and an external object range from simple, wearable and off the shelf EEG devices such as Emotiv {\bf C} and iFocusBand {\bf C} to expermental devices that physically interfaces with the brain using electrodes such as BrainGate {\bf C}. While the former is mostly used for recreational purposes, the latter is being researched and developed to give people physically disabled some means of communicating with the world. Researchers at MIT are working on a device {\bf C}, which is at a very early stage of development, that plans to fully bypass the eye and send impulses directly to the brain to give its user some kind of vision.


{\bf Prospective technology}

With the knowledge of the current innovations that are already available and the ones expected to be available to us in the near future, one may speculate as to the kind of cognitive enhancement technology that humanity may have in the far future.

In lieu with the devices of Theodore Berger {\bf C} one can imagine a range of devices that will aid and supplement the memory of human beings to beyond human levels. Science fiction writers and futurists have speculated on a device which they are calling an exocortex {\bf C}; this external cortex will supplement our current conciousness by adding more memory, processing power, system software and interfacing ports.

Due to the wide range of electical sensors that we have available, the possibilities of sensory augmentation are vast. One can augment the human vision to see a much wider range of the electromagnetic spectrum; The same can be done with hearing, tasting, smelling or even feeling. On a more imaginative and abstract note, sensory devices could be used to transpose different sensation onto different ones. If we connect our auditory devices to the visual perceptions, we may be able to see sounds, or hear the sensations of feeling and tasting.

Although any kind of cybernetic device implies some kind of interfacing between mind and machine; however here machine means an object existing outside the body. Already the exocortex mentioned earlier hinted to some kind of interfacing, however purely interfacing cybernetic devices would also be a possibility. One may have a device that just links her/his mind to the internet, or to a network of other minds using those devices. Such a device maybe used by the wearer to communicate with other devices such as his car or his 'smart' house. \\

{\bf Ethical implications}

The ethical implications of such cognitive enhancements have been listed here. In the final essay few of these issues will be focused upon and elaborated upon.

\begin{enumerate}
    \item Risk to humanbody, humanity and way of life
    \item Affordability leading to human segregation and social startification (violation of utilitarian ethics)
    %\item Different classes of humanity with different rights (rethinking of basic human rights)
    \item Communicatability between two types of sensory enhanced human
    \item Need for rethinking of most institutions; e.g. having memory chips would nullify most academic tests we have now.
    \item Manipulation of memory which maybe argued as fundamental to the fabric of human character formation (violation of virtue ethics; barrier to eudamonia)
    \item Misuse of this technology; potential harm that humanity can unleash on itself with these enhancements. This may express itself as the following (violation of duty ethics):
        \begin{itemize}
            \item Mind reading
            \item Potential hacking of other human beings
            \item Misuse by the government; e.g. with interrogation, monitoring of citizens thoughts, or altering their personality
        \end{itemize}
\end{enumerate}