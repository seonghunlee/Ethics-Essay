%Our biggest defining characteristic as human beings is our intellect. Needless to say we are not the strongest, the fastest or the most resilient to change that nature has to offer. Still we populate most of the biomes of this world and consciously define our lives and our habitats like no other animal on Earth. All of this in spite of our frailty compared to what nature has to offer, and all of this due to the only advantage that nature gave us; our ability to think better than any other animal that we know of.

The prosperity that we have achieved in the last few millennium is mainly due to our ability to make tools to complement ourselves; we have made tools that helped us adapt our environment more habitable, ensured our food supply, increased our strength and dexterity, widened our knowledge and improved our ability to carry out mental activities. However, after the  millennium of progress we are approaching a new dimension of technology; a dimension which some believe will be our salvation while others would easily relate to the legends of Icarus, Pandora’s box and Prometheus’ fire. The dimension referred to is the prospect of actively and invasively improving the human mind and body. Developments in fields such as electronics, nanotechnology, robotics, cybernetics, information technology, neurotechnology, genetic engineering and pharmacology, among others, are enabling a new field of technology labeled by some as ‘Human Engineering’ to emerge.

The authors of this paper realize that human engineering is a very broad and trans-disciplinary field, and it would take volumes to analyse its ethical implications; thus this paper attempts to narrow the field by only looking at the ethical implications of one form of ‘human engineering’; namely cybernetic human augmentation. For the purpose of this paper cybernetic human augmentation is defined as the following: any (1) electro-mechanical addition to the human body that (2) becomes a natural part of the human body and that (3) improves the performance of an individual beyond normal human capacities. Needless to say genetic or chemical forms of human augmentation are external to the purview of cybernetic augmentation.

The paper is structured as follows; the first part of the essay looks at the effects of cybernetic human augmentation on society; following this an ethical perspective is used to analyse the effects of augmentation. After looking at the effects and their ethical implications, recommendations are made to support the implementation of cybernetic augmentation on society. Finally conclusions are drawn regarding the future of humanity, its ethics and cybernetic augmentation.
